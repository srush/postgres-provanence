In this project, we extended PostgreSQL by adding native support for a provenance system which required modified postgres internals. The system we have implemented eagerly captures evolution history of tuples as various queries are run against the system. To this end, we modified the query execution engine of PostgreSQL to track provenance information up along various nodes in the query execution plan tree. We also changed the internal storage engine to store provenance information as it writes bare-minimum tuples to disk for various disk based operations (disk-based sort for example).

Through our work, we have handled a subset of SQL DDL, and support SELECTS, PROJECTS, JOINS, AGGREGATES (including GROUP BY and ORDER BY).  We however do not support stored procedures, for they are black box functions external to the database and the input data elements into the function are all that is known to the executor at run time, allowing us to capture only coarse-grained provenance information for the output data elements. 

As future work, we hope to be able to address this limitation. Measuring and profiling the CPU and storage overhead of tracking provenance in Postgres and the performance impact on the system for standard workloads would also be an interesting direction to pursue.

