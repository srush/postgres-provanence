\textbf{XXX}: outlines added for all sections, need to fill in details everywhere
% motivate provenance and briefly overview it

A large number of applications such as scientific data management, data integration, information extraction and business analytics involve processing base data to generate a large amount of new data derived from the base data. Capturing which data items contributed to creation of a new data item is important in the face of issues like questionable quality of base data, potential uncertainty associated with it, as well as the authority and trust assessment of the user performing the operations. We see that if data history is not traced in any form as the operations are performed, it would be impossible to track what future tables potential errors in early stage data collection may have propagated to. Tracking down errors and updating all tuples derived from an erroneous data item can be prohibitively costly or even impossible in the absence of this historical information.

% definition of provenance and brief overview
The term \textit{provenance} refers 
The problem of storing and quering previous data history is known as \emph{provenance}. Recent work in this area includes the Trio/LIVE system and the Panda project. However, neither of these projects handle backward provenance efficiently or provide language support for querying provenance.

% talk about lack of support in databases and current work
This lack of historical record can make database systems impractical for scientific research. 

One practical issue with available database systems is the difficulty of tracing data history. 

Our goal is to improve upon existing provenance systems, focusing on the SciDB project. The SciDB proposal \cite{stonebraker9requirements} lists three requirements for a useful provenance system -

\begin{itemize}
\item For any data element, we would like to recover the derivation history. 
\item If a data element is updated, we would like to trace forward to see other effected elements.
\item At any point, we would like to reproduce the construction of the current data. 
\end{itemize}

In particular, we hope to respond to the challenge given by \cite{cudré2009demonstration}.

\begin{quote}
Recording the log is easy. The hard part is to create a provenance query language and an efficient implementation.  
\end{quote}

Contributions:

Implemented the ability to track provenance inside postgresql

Implemented a visualizer that makes querying forward and backward provenance easier.

don't support stored procedures , they are black boxes, and unless something useful is known about the underlying functions, we anyway do not get to know too much about them.
